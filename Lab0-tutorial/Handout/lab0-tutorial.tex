\documentclass[12pt]{article}
\usepackage[margin=1in]{geometry} % change margins
\usepackage{graphicx} % inserting images
\usepackage{caption} % captions
\usepackage{amsmath} % for better math
\usepackage{verbatim} %used for comment blocks
\usepackage{mathtools}
\usepackage[english]{babel}
\usepackage[colorinlistoftodos]{todonotes}
\usepackage{placeins}
\usepackage[square, sort, numbers]{natbib}
\usepackage{url}
\usepackage{setspace}
\usepackage{breqn}
\usepackage{subcaption}
\usepackage{textcomp}
\usepackage{float}
\usepackage{leftidx}
\usepackage{bm}
\usepackage[titletoc,toc, title]{appendix}
\usepackage[utf8]{inputenc}
\usepackage{diagbox}
\usepackage{hyperref}
\usepackage{tcolorbox}

\usepackage{amsmath,amssymb,braket,cancel,nicefrac,physics,tensor,slashed}
\usepackage{color,empheq,enumitem,forloop,graphicx,microtype,subcaption,verbatim,wrapfig,pdfpages,titlesec}
\usepackage{array,booktabs,multicol,multirow,tabularx} 

\usepackage{fancyhdr}
\addto\captionsenglish{\renewcommand{\chaptername}{Lab}}

\def\cesar#1{{\color{blue}[#1]}}
\def\anhkhoi#1{{\color{olive}[#1]}}
\def\nik#1{{\color{red} [#1]}}

\tcbset{colback=blue!5!white,colframe=blue!75!black,center,halign=justify,width=0.9\textwidth,fonttitle=\bfseries}

\linespread{1.15}
\title{Tutorial: Data Analysis Practice}
\author{Nikolas Provatas, Anh-Khoi Trinh, Cesar Daniel Rodriguez Rosenblueth}
\date{}

\begin{document}

\maketitle

\section{Learning objectives}
\begin{itemize}
\item Learn how to use Excel.
\item Compute certain basic mathematical statistics.
\item Introduction to data analysis.
\end{itemize}

\section{Introduction}
This lab will be an interactive demo lab where teaching Assistants will help you  learn how to use Excel and to apply it to the concepts presented in the \textbf{Data analysis} chapter. We recommend that you read the \textbf{Data Analysis} and familiarize yourself to Excel before attending this session. The first part is a mandatory tutorial session and will be scheduled in the Lab information document on MyCourses.

\section{In-class tutorial}
During the in-class tutorial session, you will familiarize yourself with Excel. You will be guided through this exercise by the TAs.

\begin{itemize}
\item Open the Excel document found at \anhkhoi{location}.
\item Explore the user interface. Note that data is best presented in columns.
\item Learn how to resize, split and merge cells.
\item In a adjacent column, calculate the resistance $R$. 
\item Calculate the sum of resistances $R$ by using  \verb|=SUM(x)|.
\item Calculate the average resistance $\bar{R}$ in two ways: by using the integrated function  \verb|=AVERAGE(x)| and by dividing the sum that you calculated previously by the total number of cells.
\item For each value of resistance $R_i$, calculate the difference
\begin{equation}
R_i - \bar{R},
\end{equation}
in a separate column by ``dragging" the first cell downwards.

\item Calculated the square of the difference calculated above in another column.
\item In two other columns, calculate the standard deviation by using eq.~\eqref{Eq:STDev.P} with the values that you calculated previously, and by using the function \verb|=STDEV.P(x)|. 
\end{itemize}

\noindent Now we will learn how to plot data, and how to obtain a linear regression.
\begin{itemize}
\item Select the columns of $I$ and $V$, perform a scatter plot.
\item Put in manual error bars.
\item Display a linear trendline with the fit equation and its $R^2$ value. The fit coefficient is your ``fit resistance" $R_{fit}$.
\end{itemize}

\noindent Now we will compare this fit equation to our data. 
\begin{itemize}
\item To start, copy the $I$ column to another area of your spreadsheet.
\item In the column next to it, write the fit equation such that this will be your ``Fit $V$" column.
\item Plot $V$ vs $I$, and ``Fit $V$" vs $I$ on the same graph. Add error bars only on the $V$ vs $I$ dataset and compare the two graphs qualitatively.
\end{itemize}

\noindent Let us now compare resistances.
\begin{itemize}
\item On another area of your spreadsheet, write a column with the current values $I$. 
\item In an adjacent column, rewrite the resistance $R_i$ calculated previously with its corresponding standard deviation.
\item In another column, write the average resistance $\bar{R}$ for all values of $I$.
\item In another column, write the fit resistance $R_{fit}$ for all values of $I$.
\item Plot $R$ vs $I$ for all three datasets: the measurement $R_i$, the mean $\bar{R}$ and the fit $R_{fit}$. Compare them.
\end{itemize}

\section{Homework Exercise}
As homework, you will analyse a given dataset. Consider a setup where you can measure the electric force between two charges as shown in Fig.~\ref{Fig:lab0-session2-setup}. 
\begin{figure}[h]
\centering
\includegraphics[width=0.8\textwidth]{lab0-session2}
\caption{Setup for the data of Session 2. Two charges $q_1$ and $q_2$ are separated by a distance $r$, and at an angle $\theta$ with respect to some axis horizontal to $q_1$.}
\label{Fig:lab0-session2-setup}
\end{figure}

Given this dataset, you must find how the electric force on $q_1$ is affected by the following parameters:
\begin{itemize}
\item Charges $q_1$ and $q_2$.
\item Distance $r$ between the charges.
\item The angle $\theta$ with respect to a horizontal axis separating $q_1$ from $q_2$.
\end{itemize}

You are given a .txt files from which you are to import them into Excel. For all the .txt files, the charges are in Coulomb units, the forces are in Newton, the angle is in radians and the distance is in meters. \anhkhoi{Change .txt files to have the constant variables and the standard deviations.}

We suggest that you start your analysis for fixed $r, \theta$, but varying $q_1$ and $q_2$. Find how the force $F$ depends on $q_1$ and $q_2$. Support your conclusion by performing appropriate fits to your equation and comparing them to the error bars. If you find that you can simplify their dependence by 
\begin{equation}
f(x,y) = x \times y,
\label{Eq:f=xy}
\end{equation}
or 
\begin{equation}
f(x) = \frac{1}{x},
\label{Eq:f=1/x}
\end{equation} 
perform the appropriate error propagation equation to find the effective error of this composite quantity.

After you have found the appropriate $q_1$ and $q_2$ dependence, find the dependence on $\theta$ by using appropriate data analysis techniques.

Finally, find the appropriate dependence on $r$. Note that you have 3 different trials for this experiment: you must analyse all three and compare your findings to see if your analysis is self-consistent.

After analysing the effect of the all previous variables, state the equation of the electric force $F(q_1,q_2,r,\theta)$ that best matches your findings. 

Given your equation, what happens if the radius $r$ is much larger or smaller than $q1$ and $q_2$? Is this what you expect physically to occur?

Your methodology can only find the dependence on the explicit variables that you analyse, but the physics can be shifted by an overall constant:
\begin{equation}
F(q_1, q_2, r, \theta) = k \times  f(q_1,q_2,r,\theta) +c.
\end{equation}
What is the significance of $c$? Given all your previous datasets, find the constant factor $k$. If you found the right dependence on your variables, this constant should be the same for all your datasets.

\end{document}